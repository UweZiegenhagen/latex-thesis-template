\usepackage[utf8]{inputenc}
\usepackage[T1]{fontenc}
\usepackage{babel}
\usepackage{todonotes}
\usepackage{xcolor}
\usepackage{graphicx}
\usepackage{booktabs}
\usepackage{xspace}
\usepackage{subfig}
\usepackage{wrapfig}
\usepackage{paralist}
\usepackage{prettyref}


\usepackage[a4paper,left=2cm,right=3cm,top=2cm,bottom=3cm]{geometry}


% Nur wenn ein Subversion genutzt wird
\usepackage[]{svn-multi}

\newrefformat{eq}{\textup{(\ref{#1})}}
\newrefformat{lem}{Lemma \ref{#1}}
\newrefformat{thm}{Theorem \ref{#1}}
\newrefformat{cha}{Kapitel \ref{#1}}
\newrefformat{sec}{Abschnitt \ref{#1}}
\newrefformat{tab}{Tabelle \ref{#1}}
\newrefformat{fig}{Abbildung \ref{#1}}

\usepackage[]{longtable}
\usepackage[]{pdfpages}
\usepackage[]{setspace}
\onehalfspacing

\usepackage[]{microtype}
\usepackage[style=authoryear,hyperref=true,natbib=true,sorting=nyt,block=space]{biblatex}
\usepackage[babel,german=quotes]{csquotes}
\usepackage[]{listings}

\setlength{\parindent}{0pt}
\setlength{\parskip}{1.2em}

%\renewcommand{\familydefault}{\sfdefault}
%\usepackage[scaled=0.9]{helvet}

\usepackage{palatino}
%\usepackage{kpfonts}
%\usepackage{charter}
%\usepackage{beraserif}
%\usepackage{newcent}
%\usepackage{fourier}
%\usepackage{chancery}
	
\setkomafont{title}{\rmfamily} 
\setkomafont{chapter}{\huge\rmfamily}
\setkomafont{section}{\LARGE\rmfamily} 	
\setkomafont{subsection}{\Large\rmfamily} 	
\setkomafont{subsubsection}{\large\rmfamily} 	

% http://my.opera.com/timomeinen/blog/show.dml/68644
\usepackage[german]{nomencl} 
\makenomenclature 
\usepackage{makeidx}
\makeindex 
\usepackage{glossaries}
\makeglossaries
\usepackage{lineno}
\linenumbers

\usepackage[noindentafter]{titlesec}
\usepackage[flushmargin]{footmisc}

\usepackage[]{url}
\urlstyle{same}

\usepackage{hyperref}
\hypersetup{%
  colorlinks=true,   % aktiviert farbige Referenzen
  linkcolor = blue,  % Linkfarbe blau
  citecolor = blue,  % cite-Farbe blau
  urlcolor = blue,  % cite-Farbe blau
  pdfpagemode=UseNone,  % PDF-Viewer startet ohne Inhaltsverzeichnis et.al.
  pdfstartview=FitH} % PDF-Viewer benutzt beim Start bestimmte Seitenbreite

\definecolor{hellgelb}{rgb}{1,1,0.8}
\definecolor{colKeys}{rgb}{0,0,1}
\definecolor{colIdentifier}{rgb}{0,0,0}
\definecolor{colComments}{rgb}{1,0,0}
\definecolor{colString}{rgb}{0,0.5,0}

\lstset{inputencoding=utf8,%
    float=hbp,%
    language=[LaTeX]TeX,
    basicstyle=\ttfamily\small, %
    identifierstyle=\color{colIdentifier}, %
    keywordstyle=\color{colKeys}, %
    stringstyle=\color{colString}, %
    commentstyle=\color{colComments}, %
    columns=flexible, %
    tabsize=2, %
    frame=single, %
    extendedchars=true, %
    showspaces=false, %
    showstringspaces=false, %
    numbers=left, %
    numberstyle=\tiny, %
    breaklines=true, %
    backgroundcolor=\color{hellgelb}, %
    breakautoindent=true, %
    captionpos=b,%
    literate={ä}{{\"a}}1{ö}{{\"o}}1{ü}{{\"u}}1{ß}{{\ss}}1
}




\usepackage[automark]{scrpage2}
\pagestyle{scrheadings}
\clearscrheadfoot

\setheadsepline[\textwidth]{1pt}{}
\ohead{\headmark}
\ofoot[\pagemark]{\pagemark}
\cfoot{}
\chead{}